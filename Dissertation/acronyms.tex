\newacronym{dbms}{СУБД}{Система управления базами данных}
\newacronym{db}{БД}{База данных}
\newacronym{hep}{ФВЭ}{Физика высоких энергий}
\newacronym{lms}{МНК}{Метод наименьших квадратов}
\newacronym{oop}{ООП}{Объектно-ориентированное программирование}
\newacronym{dsl}{ПОЯ}{Предметно-ориентированный язык программирования}
\newacronym{sw}{ПО}{Программное обеспечение}
\newacronym{api}{API}{\emph{англ.} Application program interface, программный интерфейс приложения}
\newacronym{fpga}{ПЛИС}{Программируемая логическая интегральная схема}
\newacronym{sfinae}{SFINAE}{Substitution failure is not an error}
\newacronym{crtp}{CRTP}{Curiously recurring template pattern}
\newacronym{fdd}{FDD}{Feature-driven development}
\newacronym{ddd}{DDD}{Domain-driven development}
\newacronym{htc}{HTC}{High throughput computing}
\newacronym{hpc}{HPC}{High performance computing}
\newacronym{mc}{МК}{методы моделирования Монте-Карло}

\newacronym{adc}{АЦП}{Аналогово-цифровой преобразователь}
\newacronym{sadc}{САЦП}{Сэмплирующий аналогово-цифровой преобразователь}
\newacronym{tdc}{ВЦП}{Время-цифровой преобразователь}
\newacronym{pmt}{ФЭУ}{Фотоэлектронный умножитель}

\newacronym{sm}{СМ}{Стандартная модель}
\newacronym{mip}{MIP}{Minimum ionizing particle, минимально-ионизирующая частица}
\newacronym{dm}{ТМ}{Тёмная материя}
