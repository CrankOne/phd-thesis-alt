%%%%%%%%%%%%%%%%%%%%%%%%%%%%%%%%%%%%%%%%%%%%%%%%%%%%%%
%%%% Файл упрощённых настроек шаблона диссертации %%%%
%%%%%%%%%%%%%%%%%%%%%%%%%%%%%%%%%%%%%%%%%%%%%%%%%%%%%%

%%% Инициализирование переменных, не трогать!  %%%
\newcounter{intvl}
\newcounter{otstup}
\newcounter{contnumeq}
\newcounter{contnumfig}
\newcounter{contnumtab}
\newcounter{pgnum}
\newcounter{chapstyle}
\newcounter{headingdelim}
\newcounter{headingalign}
\newcounter{headingsize}
%%%%%%%%%%%%%%%%%%%%%%%%%%%%%%%%%%%%%%%%%%%%%%%%%%%%%%

%%% Область упрощённого управления оформлением %%%

%% Интервал между заголовками и между заголовком и текстом %%
% Заголовки отделяют от текста сверху и снизу
% тремя интервалами (ГОСТ Р 7.0.11-2011, 5.3.5)
\setcounter{intvl}{3}               % Коэффициент кратности к размеру шрифта

%% Отступы у заголовков в тексте %%
\setcounter{otstup}{0}              % 0 --- без отступа; 1 --- абзацный отступ

%% Нумерация формул, таблиц и рисунков %%
% Нумерация формул
\setcounter{contnumeq}{0}   % 0 --- пораздельно (во введении подряд,
                            %       без номера раздела);
                            % 1 --- сквозная нумерация по всей диссертации
% Нумерация рисунков
\setcounter{contnumfig}{0}  % 0 --- пораздельно (во введении подряд,
                            %       без номера раздела);
                            % 1 --- сквозная нумерация по всей диссертации
% Нумерация таблиц
\setcounter{contnumtab}{1}  % 0 --- пораздельно (во введении подряд,
                            %       без номера раздела);
                            % 1 --- сквозная нумерация по всей диссертации

%% Оглавление %%
\setcounter{pgnum}{1}       % 0 --- номера страниц никак не обозначены;
                            % 1 --- Стр. над номерами страниц (дважды
                            %       компилировать после изменения настройки)
\settocdepth{subsection}    % до какого уровня подразделов выносить в оглавление
\setsecnumdepth{subsection} % до какого уровня нумеровать подразделы


%% Текст и форматирование заголовков %%
\setcounter{chapstyle}{1}     % 0 --- разделы только под номером;
                              % 1 --- разделы с названием "Глава" перед номером
\setcounter{headingdelim}{1}  % 0 --- номер отделен пропуском в 1em или \quad;
                              % 1 --- номера разделов и приложений отделены
                              %       точкой с пробелом, подразделы пропуском
                              %       без точки;
                              % 2 --- номера разделов, подразделов и приложений
                              %       отделены точкой с пробелом.

%% Выравнивание заголовков в тексте %%
\setcounter{headingalign}{0}  % 0 --- по центру;
                              % 1 --- по левому краю

%% Размеры заголовков в тексте %%
\setcounter{headingsize}{0}   % 0 --- по ГОСТ, все всегда 14 пт;
                              % 1 --- пропорционально изменяющийся размер
                              %       в зависимости от базового шрифта

%% Подпись таблиц %%

% Смещение строк подписи после первой строки
\newcommand{\tabindent}{0cm}

% Тип форматирования заголовка таблицы:
% plain --- название и текст в одной строке
% split --- название и текст в разных строках
\newcommand{\tabformat}{plain}

%%% Настройки форматирования таблицы `plain`

% Выравнивание по центру подписи, состоящей из одной строки:
% true  --- выравнивать
% false --- не выравнивать
\newcommand{\tabsinglecenter}{false}

% Выравнивание подписи таблиц:
% justified   --- выравнивать как обычный текст («по ширине»)
% centering   --- выравнивать по центру
% centerlast  --- выравнивать по центру только последнюю строку
% centerfirst --- выравнивать по центру только первую строку (не рекомендуется)
% raggedleft  --- выравнивать по правому краю
% raggedright --- выравнивать по левому краю
\newcommand{\tabjust}{justified}

% Разделитель записи «Таблица #» и названия таблицы
\newcommand{\tablabelsep}{~\cyrdash\ }

%%% Настройки форматирования таблицы `split`

% Положение названия таблицы:
% \centering   --- выравнивать по центру
% \raggedleft  --- выравнивать по правому краю
% \raggedright --- выравнивать по левому краю
\newcommand{\splitformatlabel}{\raggedleft}

% Положение текста подписи:
% \centering   --- выравнивать по центру
% \raggedleft  --- выравнивать по правому краю
% \raggedright --- выравнивать по левому краю
\newcommand{\splitformattext}{\raggedright}

%% Подпись рисунков %%
%Разделитель записи «Рисунок #» и названия рисунка
\newcommand{\figlabelsep}{~\cyrdash\ }  % (ГОСТ 2.105, 4.3.1)
                                        % "--- здесь не работает

%%% Цвета гиперссылок %%%
% Latex color definitions: http://latexcolor.com/
\definecolor{linkcolor}{rgb}{0.9,0,0}
\definecolor{citecolor}{rgb}{0,0.6,0}
\definecolor{urlcolor}{rgb}{0,0,1}
%\definecolor{linkcolor}{rgb}{0,0,0} %black
%\definecolor{citecolor}{rgb}{0,0,0} %black
%\definecolor{urlcolor}{rgb}{0,0,0} %black

%                                 ________________________
% ______________________________/ User-defined glossaries

\usepackage[acronym,nopostdot,nonumberlist]{glossaries}
% nopostdot -- do not use dot after the article
% nonumberlist -- do not print list of pages
%
% Custom Glossary Style {{{
\newglossarystyle{plainemdash}{
  \setglossarystyle{list}
  \renewenvironment{theglossary}
    {\begin{list}{}{\setlength{\leftmargin}{0pt}
                    \setlength{\labelwidth}{0pt}
                    \setlength{\itemindent}{6.75pt}
                    \setlength{\listparindent}{0pt}
                    \setlength{\parsep}{\parskip}}}
    {\end{list}}
    \renewcommand*{\glossentry}[2]{%
        \item[\glsentryitem{##1}\glstarget{##1}{\textbf{\glossentryname{##1}}}]%
        ---\ %
        \glossentrydesc{##1}%
        \unskip\leaders\hbox to 2.9mm{\hss.}\hfill##2}%
}
\makeglossaries
\newacronym{dbms}{СУБД}{Система управления базами данных}
\newacronym{db}{БД}{База данных}
\newacronym{hep}{ФВЭ}{Физика высоких энергий}
\newacronym{lms}{МНК}{Метод наименьших квадратов}
\newacronym{oop}{ООП}{Объектно-ориентированное программирование}
\newacronym{dsl}{ПОЯ}{Предметно-ориентированный язык программирования}
\newacronym{sw}{ПО}{Программное обеспечение}
\newacronym{api}{API}{\emph{англ.} Application program interface, программный интерфейс приложения}
\newacronym{fpga}{ПЛИС}{Программируемая логическая интегральная схема}
\newacronym{sfinae}{SFINAE}{Substitution failure is not an error}
\newacronym{crtp}{CRTP}{Curiously recurring template pattern}
\newacronym{fdd}{FDD}{Feature-driven development}
\newacronym{ddd}{DDD}{Domain-driven development}
\newacronym{htc}{HTC}{High throughput computing}
\newacronym{hpc}{HPC}{High performance computing}
\newacronym{mc}{МК}{методы моделирования Монте-Карло}

\newacronym{adc}{АЦП}{Аналогово-цифровой преобразователь}
\newacronym{sadc}{САЦП}{Сэмплирующий аналогово-цифровой преобразователь}
\newacronym{tdc}{ВЦП}{Время-цифровой преобразователь}
\newacronym{pmt}{ФЭУ}{Фотоэлектронный умножитель}
\newacronym{pmma}{ПММА}{Полиметилметакрилат}

\newacronym{sm}{СМ}{Стандартная модель}
\newacronym{mip}{MIP}{Minimum ionizing particle, минимально-ионизирующая частица}
\newacronym{dm}{ТМ}{Тёмная материя}
\newacronym{daf}{DAF}{\emph{англ.} Deterministic annealing filter}
\newacronym{krf}{KRF}{Kalman filter with reference track}

\newacronym{sle}{СЛАУ}{Система линейных уравнений}

\newglossaryentry{serialization}
{
    name=сериализация,
    description={представление экземпляра составного типа данных в виде
    байтовой последовательности, используемое для хранения или
    передачи данных по сетевым протоколам, используемое с целью последующего
    восстановления экземпляра второй стороной (десереализации)},
    %
    user1=сериализации, % родительный, (нет) кого, чего?
    user2=сериализации, % дательный (дам) кому? чему?
    user3=сериализацию, % винительный (вижу) кого? что?
    user4=сериализацией, % творительный (доволен) кем? чем?
    user5=сериализации, % предложный (думаю) о ком? о чём?
}

\newglossaryentry{deserialization}
{
    name=десериализация,
    description={восстановление экземпляра составного типа данных из
    байтовой последовательности},
    user1=десериализации, % родительный, (нет) кого, чего?
    user2=десериализации, % дательный (дам) кому? чему?
    user3=десериализацию, % винительный (вижу) кого? что?
    user4=десериализацией, % творительный (доволен) кем? чем?
    user5=десериализации, % предложный (думаю) о ком? о чём?
}

\newglossaryentry{allocator}
{
    name=аллокатор,
    description={объект управляющий динамическим выделением и освобождением памяти},
    user1=аллокатора, % родительный, (нет) кого, чего?
    user2=аллокатору, % дательный (дам) кому? чему?
    user3=аллокатор, % винительный (вижу) кого? что?
    user4=аллокатором, % творительный (доволен) кем? чем?
    user5=аллокаторе, % предложный (думаю) о ком? о чём?
}

\newglossaryentry{metaprogramming}{
    name=метапрограммирование,
    description={подход к разработке программного обеспечения, при
    котором создаются программы способные генерировать или преобразовывать
    другие программы (в том числе самих себя) как данные},
    user1=метапрограммирования, % родительный, (нет) кого, чего?
    user2=метапрограммированию, % дательный (дам) кому? чему?
    user3=метапрограммирование, % винительный (вижу) кого? что?
    user4=метапрограммированием, % творительный (доволен) кем? чем?
    user5=метапрограммировании, % предложный (думаю) о ком? о чём?
}

\newcommand{\glsgenitive}[1]{\glsuseri{#1}}   % ...... родительный, (нет) кого, чего?
\newcommand{\glsdative}[1]{\glsuserii{#1}}  % ......... дательный (дам) кому? чему?
\newcommand{\glsaccusative}[1]{\glsuseriii{#1}}  % .... винительный (вижу) кого? что?
\newcommand{\glsinstrumental}[1]{\glsuseriv{#1}} % .... творительный(доволен) кем? чем?
\newcommand{\glsprepositional}[1]{\glsuserv{#1}}  % ... предложный (думаю) о ком? о чём?

%    user1=,  % родительный, (нет) кого, чего?
%    user2=,  % дательный (дам) кому? чему?
%    user3=,  % винительный (вижу) кого? что?
%    user4=,  % творительный (доволен) кем? чем?
%    user5=,  % предложный (думаю) о ком? о чём?

%\textbf{TeX} : Cистема компьютерной вёрстки, разработанная американским профессором информатики Дональдом Кнутом
%\textbf{панграмма} : Короткий текст, использующий все или почти все буквы алфавита

