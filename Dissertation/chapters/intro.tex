\chapter*{\centering Введение}
\addcontentsline{toc}{chapter}{Введение}

\section*{Актуальность работы}

Экспериментальная физика частиц ставит своей целью наблюдение конкретных
физических процессов с использованием прямых или косвенных методов
измерения. В современных экспериментах прямые измерения осуществляются
при помощи цифровой измерительной аппаратуры, включающей быстродействующие
вычислительные системы и программируемые логические устройства.
Полученные в ходе экспериментов данные сохраняются на цифровых носителях
с целью их долговременного хранения и последующего анализа. Алгоритмическая
обработка и физический анализ этих данных осуществляется с использованием
программного обеспечения (\acrshort{sw}), реализованного преимущественно на
языках программирования высокого уровня и обладающих широким спектром
выразительных средств для описания структур данных и операций над ними.

Жизненный цикл научного \acrshort{sw}, как
правило, существенно короче, чем у коммерческих программных
продуктов~\cite{lifecycle-lenhardt2014data}.
По завершении измерений, проверке гипотез и публикации
результатов (то есть после достижения целей исследовательской
программы) программный комплекс как целое быстро утрачивает актуальность
вследствие эволюции методов измерений, инструментальных средств анализа,
изменений в теоретических моделях.
Тем не менее отдельные компоненты таких систем сохраняют
научную и техническую ценность и, следовательно, должны быть
спроектированы с возможностью повторного использования, что
подразумевает требования изолированности и переносимости этих
компонент~\cite{sci-software-eng-arvanitou2021, software-for-science-CarverEtAl2016, Kaida2023-architecture}.

В то же время, в условиях отсутствия чёткой архитектурной
стратегии, естественным образом происходит спонтанное формирование
программной экосистемы, когда отдельные разработчики реализуют
решения исходя из собственных локальных нужд. Несмотря на стремление к
устойчивому развитию программных систем для экспериментальной физики,
выражающееся в попытках создания различных общих и частных
решений~\cite{gaudi-framework-1, geant4-agostinelli, ROOT-framework, hep-roadmap-Albrecht2019},
локальность задач экспериментальной физики и их распределённый
характер часто принципиально не позволяют выработать единую архитектуру
на начальных этапах. Каждая экспериментальная коллаборация вырабатывает свою
программную базу индивидуально, что нередко приводит к
дублированию функциональности и усложняет достижение устойчивости
экосистемы~\cite{elmer2018strategicplanscientificsoftware}. Тем не менее,
такой подход уже длительное время является преобладающим,
и упоминается в литературе (ad hoc или <<bottom-up подход>>)
как историческое сопротивление структурированным практикам
разработки~\cite{frameworks-Brun2011}: <<от ситуативных решений к
повторно-используемым системам>>. В результате, попытки внедрения уже
реализованных решений в другие задачи становятся менее оправданными, чем
их повторная реализация. В работах посвящённых разработке программного
обеспечения для научных задач вообще~\cite{software-for-science-CarverEtAl2016, lifecycle-lenhardt2014data},
и для задач автоматизации физического эксперимента в частности~\cite{Przedzinski2020PhD},
подчёркиваются отличия от промышленной разработки, которые
требуют глубокого погружения в предметную область, что существенно
их применимость.

Так например, в последние десятилетия значительное внимание в физике высоких
энергий уделяется экспериментальным постановкам, направленным на
поиск редких процессов и слабо взаимодействующих частиц, не
охватываемых рамками Стандартной модели~(\acrshort{sm}). Одним из наиболее
распространённых классов подобных экспериментов являются
установки типа <<\emph{beam dump}>>~(герметичный сброс пучка).
В этих постановках высокоэнергетический пучок частиц
направляется на массивную мишень с высоким атомным номером поглотителя,
которая поглощает практически всю энергию первичных частиц описываемых
\acrshort{sm}. При этом возможно образование гипотетических слабосвязанных
объектов -- лёгких бозонов, стерильных нейтрино, аксионоподобных
частиц, которые способны покидать материал мишени за счёт малых сечений
взаимодействия. Благодаря экранированию интенсивного фонового
излучения, за экранирующей мишенью возможно разместить чувствительные
детекторы, позволяющие либо регистрировать продукты распада гипотетических
частиц, либо логически исключать редкие процессы утечки за счёт
слабовзаимодействующей нейтральной компоненты излучения.

Наряду с этим в последние годы в литературе активно обсуждается
другая похожая методика -- <<\emph{light shining through the wall}>>
(<<свет сквозь стену>>), основанная на конверсии частиц \acrshort{sm}
в слабовзаимодействующие частицы (чаще всего -- лёгкие бозоны), и
последующей регистрации продуктов их распада за непрозрачным
барьером (разница, таким образом, заключается в основном в том где
ожидается образование гипотетических частиц).

Эксперименты в физике высоких энергий в целом требуют
анализа большого объёма экспериментальной статистики, а в случае
перечисленных методик, нуждаются в надёжных средствах автоматизации вычислений,
позволяющих гибко организовать процесс разработки сложных алгоритмов
реконструкции за счёт модульного подхода проектирования и повышения доли
использования уже реализованных решений.

Так например, в настоящее время наиболее признанной теоретической конструкцией,
описывающей природу элементарных частиц и их фундаментальных
взаимодействий, является Стандартная модель (СМ)~\cite{standard-model-maioli}.
Её формализм представляет собой строго определённую систему, объединяющую
три из четырёх известных типов фундаментальных взаимодействий. При
этом гравитационное взаимодействие, а также широкий класс феноменов,
связанных, например, с составом и ускоренным расширением Вселенной,
не входят в область применимости СМ и находятся за пределами её
эпистемологических границ.

%В частности, в 2019 году Нобелевскую премию по физике получил Джеймс Пиблс,
%среди отмеченных премией работ которого была, интерпретация гранулярности
%реликтового излучения \cite{peebles.fluctuations} на основании данных
%спутниковой обсерватории Planck \cite{planck.2018.results}. Эти данные
%наглядно демонстрируют три пика соответствующих
%основным акустическим модам конденсации первичной материи после
%Большого взрыва позволяют получить оценки коэффициентов в уравнении
%Фридмана и сделать количественную оценку массового вклада небарионной
%материи -- так называемой \emph{тёмной материи}. Это постулируемая
%форма материи, которая не взаимодействует электромагнитно и, следовательно,
%является невидимой с точки зрения Стандартной Модели (СМ) элементарных
%частиц, но участвует в гравитационных взаимодействиях. 
%Наряду с ТМ мы имеем дело также с темной энергией (ТЭ).  
%В космологии и астрономии ТЭ — это предполагаемая форма энергии,
%воздействующая на Вселенную на больших масштабах. В частности, её
%основной эффект заключается в ускоренном расширении Вселенной. 
%Согласно оценкам современных космологических моделей 
%(Лямбда-CDM) до 95\% массы вещества во Вселенной содержится 
%в виде ТМ (23\%) и ТЭ (72\%).
Значимость вопросов, связанных с космологическими аспектами, была
подтверждена присуждением Нобелевской премии по физике Джеймсу
Пиблсу в 2019 году. В числе отмеченных его работ особое место
занимает интерпретация гранулярной структуры
реликтового излучения~\cite{peebles.fluctuations}, выполненная
на основе данных спутниковой обсерватории Planck~\cite{planck.2018.results}.
Анализ этих данных позволил выявить три выраженных пика,
соответствующих основным акустическим модам конденсации первичной
материи после Большого взрыва. Данный результат открыл возможность
для количественного определения параметров уравнения Фридмана и,
в частности, для оценки массового вклада небарионной компоненты
материи, обозначаемой как тёмная материя (\acrshort{dm}). Под последней понимается
гипотетическая форма материи, не участвующая в электромагнитных
взаимодействиях и, следовательно, слабо доступная для прямого наблюдения
средствами Стандартной модели, но проявляющая себя в
гравитационных эффектах.

Наряду с тёмной материей в космологических моделях выделяется
также тёмная энергия -- постулируемая форма энергии, действующей
на масштабах Вселенной и обусловливающей наблюдаемое ускоренное
космологическое расширение. Согласно современным представлениям,
отражённым в рамках модели Лямбда-CDM ($\Lambda$ Cold Dark
Matter)~\cite{lambda-cdm}, до 95\%
массы–энергии Вселенной приходится на невидимые компоненты:
около 23\% на тёмную материю и порядка 72\% на тёмную энергию.

%Концепция тёмной материи исторически связана с проблемой скрытой массы,
%когда наблюдаемое движение небесных тел отклоняется от законов небесной
%механики. Как правило, это явление объяснялось существованием
%неизвестного материального тела (или нескольких тел). Так были открыты
%планета Нептун и звезда Сириус.
%Сам термин <<тёмная материя>>, вероятно, был впервые введён в 1906
%году Анри Пуанкаре, развивавшим идеи Кельвина об оценке массы звёзд
%Галактики на основе распределения их скоростей.
Концепция тёмной материи восходит к проблеме <<скрытой массы>>,
проявляющейся в несоответствии наблюдаемого движения небесных тел
предсказаниям ньютоновской механики~\cite{hubble-distance}.
Подобные аномалии традиционно
интерпретировались как следствие гравитационного воздействия ещё
не открытых тел. Именно таким образом были обнаружены планета
Нептун и звезда Сириус. Сам термин <<тёмная материя>>, по-видимому,
впервые был предложен Анри Пуанкаре в 1906 году в контексте развития
идей У. Томсона (Кельвина), связанных с оценкой массы звёзд Галактики
на основании распределения их скоростей~\cite{bertone-history-of-dark-matter}.

%Многочисленные косвенные свидетельства в астрофизике, космологии и физике частиц
%указывают на существование ТМ и реакций конверсии между барионной материей и ТМ.
%Помимо упомянутой диаграммы в работах Пибблса,
%например, спутниковый детектор PAMELA обнаружил аномальную насыщенность позитронами
%состава космических лучей в диапазоне энергий 5-100~ГэВ, со значительным
%увеличением количества позитронов в жёсткой части спектра. В качестве возможных
%объяснений авторы \cite{PAMELA.Adriani2009} рассматривают как неучтённые
%первичные источники (т.н. микроквазары), ускорение заряженных частиц
%магнитосферой пульсаров (в этом случае, однако, должны соответственно
%изменяться и спектры для электронов), так и проявление феноменов не
%учтённых Стандартной моделью -- присутствие короткоживущих
%слабо-взаимодействующих частиц, способных распадаться в частицы барионной
%материи. Ссылаясь на свои прежние публикации, авторы указывают на отсутствие
%аналогичного избытка в протон-антипротонных спектрах космических лучей и
%заключают, что распад происходит в основном на лептонные пары.
В последующем в астрофизике, космологии и физике элементарных частиц
накопилось значительное число косвенных свидетельств существования
тёмной материи и процессов конверсии между ней и барионным веществом.
Помимо анализа акустических осцилляций первичной плазмы, существенный
интерес представляют результаты космических экспериментов.
Так, спутниковый детектор PAMELA~\cite{PAMELA.Adriani2009}
зарегистрировал аномальный избыток позитронов
в космических лучах в энергетическом диапазоне 5-100 ГэВ, проявляющееся
резким возрастанием их доли в жёсткой части спектра. В качестве
возможных объяснений этого эффекта обсуждались как неучтённые
астрофизические источники (например, микроквазары), так и
ускорение заряженных частиц в магнитосферах пульсаров, что должно
было бы, однако, сопровождаться также модификацией спектра электронов.
Рассматривался также сценарий, предполагающий существование слабо
взаимодействующих частиц за пределами Стандартной модели, способных
распадаться с образованием барионных продуктов~\cite{PAMELA.Adriani2009}.
Дополнительно, гипотеза об ускорении позитронов в магнитосферах квазаров
была детально рассмотрена в работе~\cite{malyshev.vs.review}. Показано, что
в случае подобного происхождения высокоэнергетической позитронной
компоненты спектр космических лучей должен содержать особенности,
которые не наблюдаются в данных экспериментов PAMELA~\cite{PAMELA.Adriani2009}
и ATIC~\cite{malyshev.vs.review}.

Более общий подход к согласованию теоретических оценок реликтовой
плотности тёмной материи с астрофизическими наблюдениями заключается
во введении так называемых портальных взаимодействий, посредством
которых осуществляется связь между видимым и тёмным секторами через
специальные медиаторы. Наиболее распространённым сценарием является
механизм вымораживания (англ. \emph{freeze-out}~\cite{freezout-Krnjaic}),
при котором частицы
тёмной материи аннигилируют в частицы Стандартной модели через обмен
гипотетическими массивными медиаторами. В качестве таких медиаторов
в литературе рассматриваются различные спиновые состояния:
скалярные~\cite{scalars-mcdonald},
векторные~\cite{okun1092limits},
псевдоскалярные~\cite{axion-portal-nomura, tsai.axion},
тензорные~\cite{Döbrich2016},
а также фермионные~\cite{nasri-fermion}.

%Следует подчеркнуть, что помимо астрофизических и космологических
%наблюдений, гипотезы о природе и свойствах тёмной материи используются
%в физике элементарных частиц для объяснения ряда других нерешённых
%фундаментальных проблем. К ним, в частности, относятся вопрос о барионной
%асимметрии Вселенной (связанной с нарушением комбинированной зарядовой и %пространственной $CP$-инвариантности), аномальное значение
%магнитного момента мюона, а также наблюдаемые аномалии в редких
%распадах $B$-мезонов.
Указанные наблюдения с одной стороны, а с другой -- статистика накопленных
к настоящему времени данных на ускорителях, формируют верхние и нижние
границы для диапазонов масс и констант связи, определяющих взаимодействие
барионной материи с тёмным сектором. В то же время необходимо
отметить, что гипотезы о природе и механизмах возникновения тёмной
материи имеют более широкий контекст: в физике элементарных частиц они
используются для объяснения таких фундаментальных проблем,
как происхождение барионной асимметрии Вселенной (связанной с
нарушением комбинированной зарядовой и пространственной
$CP$-инвариантности~\cite{babar-cp-b-meson, Gninenko:2021}), отклонение
экспериментального значения магнитного
момента мюона от предсказаний Стандартной
модели~\cite{g-2-problem, na64-sieber-g-2-mu}, а также аномалии,
наблюдаемые в редких распадах $B$-мезонов~\cite{babar-cp-b-meson}.

%Астрофизические ограничения с одной стороны, и статистика экспериментальных
%данных, полученных на ускорителях высоких энергий, с другой стороны,
%позволяют устанавливать верхние и нижние пределы для диапазонов масс
%констант связи портальных моделей, определяющих интенсивность
%взаимодействия барионной материи с компонентами тёмного сектора.


%Гипотеза о происхождении позитронного избытка в результате ускорения в
%магнитосфере квазаров была рассматрена в работе \cite{malyshev.vs.review}.
%Показано, что в случае такого происхождения высокоэнергетической позитронной
%компоненты космических лучей, соответствующие участки спектра обязаны были бы
%иметь особенности отсутствующие в результатах PAMELA, ATIC.

% TODO: figure from Planck results 2018

Ввиду малости гипотетических сечений в реакциях образования
\acrshort{dm} (определяемых в основном $\epsilon$ и $g_{a \gamma \gamma}$),
отыскание механизмов её образования и проверка моделей может производиться либо
с использованием чрезвычайно высоких энергий взаимодействующих частиц (от сотен ТэВ), в настоящее время недоступных на ускорителях частиц практически,
либо путём накопления большой статистики прецизионных измерений реакций неупругих
взаимодействий в среднем энергетическом диапазоне (десятки и сотни ГэВ), включающих
тот или иной механизм расширения СМ. Второй подход получил
развитие в последние десятилетия в рамках экспериментов в герметичной постановке
(<<\emph{light shining through the wall}>> или <<\emph{beam dump}>>, <<\emph{active beam dump}>>). 

% TODO: численные оценки статистики, выходов, фонов

В этой связи в таких экспериментах решающее значение имеет оценка
абсолютных сечений фоновых процессов  и точность реконструкции физических
величин в смысле энергетического и временного разрешения.
