\chapter*{\centering Заключение}
\addcontentsline{toc}{chapter}{Заключение}

% 5.3.3 В заключении излагают итоги выполненного исследования,
%     рекомендации, перспективы дальнейшей разработки темы
%
% На мой взгляд, можно было бы использовать термин «эксперимент с фиксированной мишенью» в качестве обобщающего для применения разработанного ПК, добавляя в необходимых для этого местах уточнение «например, для эксперимента NA64». Упоминание этих конкретных условий и объекта станет очень уместным в соответствующих разделах диссертации, например, в разделе о практической применимости и внедрении результатов. Более того, напоминаю, что в Заключении, кроме основных выводов должны быть представлены перспективы и предложения по использованию полученных результатов – CERN и коллаборация здесь вообще «не сыграют», а скорее навредят. В Заключении автореферата сейчас перечислены, в основном, частные достижения, а они там не нужны; придётся переформулировать в обобщающие научные выводы, исключив заодно всякие «коллаборации» и прочие «уникальные условия».

Разработана и обоснована архитектура программного комплекса
предназначенного для реконструкции и анализа событий в
рамках задач возникающих в физическом эксперименте с триггерной
системой.
Предложенная архитектура реализует ограниченный набор
архитектурных инвариантов:
модульность,
потоковая обработка,
наличие объектной модели события,
детерминированность и идемпотентность вычислений.
Сведение этих принципов в программную архитектуру и их
обобщённая реализация являются оригинальным техническим
решением, при помощи которого, в рамках рассмотренной гибридной методологии
разработки, в работе на конкретных
примерах обеспечен полный цикл сопровождения эксперимента -- показаны этапы
моделирования, реконструкции, анализа и обработки данных. В частности:
\begin{enumerate}
    \item Обобщённая реализация конвейера обработки данных обеспеченного
    калибровочной информацией и усиленного набором встраиваемых
    искусственных языков, позволяет проводить реконструкцию
    и анализ экспериментальных событий, включая этапы калибровки и
    мониторинга во время набора данных,
    \item Предложенный генератор модельных событий для процесса
    фотообразования частицы $A'$ (тёмный фотон) на основе аналитических
    сечений позволяет эффективно моделировать
    класс реакций образования гипотетических частиц на тяжёлых ядрах,
    связанных со Стандартной моделью электромагнитным взаимодействием,
    \item Предложенный метод калибровки герметичных детекторов не
    нуждается в априорных оценках энерговыделения, и позволяет проводить
    перекалибровку детектора во время набора данных,
    \item Реализованная подсистема генерации машин конечных состояний для
    решения задач минимизации и численной аппроксимации позволяет динамически
    обуславливать различные сценарии отбора гипотез. В частности, обобщённая
    реализация алгоритма отыскания треков позволяет снизить комбинаторный фон,
    а алгоритм подгонки функции отклика сигналов с сэмплирующих
    амплитудно-цифровых преобразователей позволяет восстанавливать сигналы
    с использованием конкурирующих гипотез априорной формы
    импульса (частично компенсируя ограничения на частоту Найквиста).
\end{enumerate}

Практическая значимость предложенных решений:

\begin{itemize}
    \item Архитектурные инварианты и реализованные механизмы
    расширения программного комплекса нацелены на разработку
    в рамках коротких циклов, допускают упрощенную интеграцию с
    другими экспериментами построенными на триггерной логике (включающие
    иные калориметрические конфигурации, трековые подсистемы,
    системы сбора данных), снижая затраты на сопровождение
    и повторное использование программного кода, и отвечая таким образом
    основным задачам автоматизации физического эксперимента.
    \item Параметрическая реконструкция сигналов и связанная с ней
    свёрточная модель ливня повышают устойчивость извлечения
    признаков (временные и амплитудные характеристики)
    при разных частотах дискретизации и геометриях
    сэмплирующих калориметров
    \item Генератор с аналитическими сечениями ускоряет
    численные оценки фона/сигналов в сценариях поиска слабых
    сигналов по сравнению с более общими методами, не
    учитывающими форму.
\end{itemize}


%а также явная спецификация точек расширения в
%рамках обобщённых шаблонных реализаций конвейера данных.
% нельзя ли как-то связать с "гибридной методологией разработки"?

\begin{comment}
\paragraph{Благодарности}

Автор выражает благодарность научному руководителю
диссертационной работы профессору Валерию Ефимовичу Любовицкому
за знакомство с научным процессом современной физики высоких
энергий и обширную теоретическую перспективу. Автор глубоко
признателен руководителю эксперимента NA64 Сергею Николаевичу
Гниненко за предоставленные возможности и проницательные
комментарии о постановке эксперимента.
Михаила Михайловича Кирсанова за поддержку и наставления.
Автор благодарит Алексея Эдуардовича Шевелёва за неоценимую помощь
в реализации модулей реконструкции треков и тестировании всего
программного окружения. Сергея Герасимова за тестирование и ценные
комментарии об обобщённой реализации алгоритма CATS применительно
к модельным данным Belle II, Миральда Тузи (Mirald Tuzi) и
Генри Зибера (Dr. Henri Hugo Sieber) за тестирование реализации алгоритма
CATS в NA64. Решающий вклад в практическое тестирование, интеграцию и отладку
параметров реконструкции сигналов сэмплирующих преобразователей применительно
к физическим задачам внесли Андреа Челентано
(Dr. Andrea Celentano) и Лука Марсикано (Dr. Luca Marsicano).
%Автор благодарит весь коллектив коллаборации NA64 за труд
\end{comment}