\section{Калибровка детекторов}

\subsection{Калибровка детекторов просматриваемых фотоумножителями}

В предположении, что в отдельном чувствительно объёме детектора
просматриваемом \acrshort{pmt} зависимость уровня сигнала $A$ от
выделившейся в рабочем объёме детектора энергии~$E$ с хорошей
точностью аппроксимируется линейной
зависимостью~$E \simeq C \cdot A$,
калибровка соответствующего детектора состоит в отыскании
коэффициентов~$C$ для известной энергии $E$ и измеренной амплитуды
сигнала~$A$.

Эти коэффициенты трудно или невозможно посчитать заранее, поскольку
они зависят от множества технических и физических условий стационарных
лишь в некотором приближении, включающих уровень высокого напряжения
питающей системы, параметры входных и выходных каскадов элементной
базы, напряжения на динодах \acrshort{pmt}, качества оптического контакта
и т.д.
%~\acrshort{pmt} на динодной системе, а этот уровень в свою очередь выбирается таким
%образом, чтобы обеспечить достаточное энергетическое
%разрешение \acrshort{pmt} в счётном режиме работы -- обычно так,
%чтобы номинальная энергия пучка не превышала верхнюю
%границу~\acrshort{adc}. При этом, коэффициент $C$ подвержен дрейфу
%вследствие различных физических процессов в выходном каскаде
%питающей аппаратуры, в динодной системе и на фотокатоде,
%в веществе детектора.
В силу случайного характера этих процессов
обусловленных в основном индивидуальными параметрами детектора
и \acrshort{pmt}, отыскание калибровочных коэффициентов $C$
можно осуществить на основе процессов с известным спектром.
Таким эталоном могут служить, например, мюоны, для которых обычно
хорошо известен пик минимально ионизирующих частиц~(\acrshort{mip}),
атмосферные мюоны, или моноэнергетические пучки искусственного
происхождения, для которых можно получить спектральные
оценки из моделирования методами Монте-Карло.


