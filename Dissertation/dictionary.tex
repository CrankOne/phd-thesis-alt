\newglossaryentry{serialization}
{
    name=сериализация,
    description={представление экземпляра составного типа данных в виде
    байтовой последовательности, используемое для хранения или
    передачи данных по сетевым протоколам, используемое с целью последующего
    восстановления экземпляра второй стороной (десереализации)},
    %
    user1=сериализации, % родительный, (нет) кого, чего?
    user2=сериализации, % дательный (дам) кому? чему?
    user3=сериализацию, % винительный (вижу) кого? что?
    user4=сериализацией, % творительный (доволен) кем? чем?
    user5=сериализации, % предложный (думаю) о ком? о чём?
}

\newglossaryentry{deserialization}
{
    name=десериализация,
    description={восстановление экземпляра составного типа данных из
    байтовой последовательности},
    user1=десериализации, % родительный, (нет) кого, чего?
    user2=десериализации, % дательный (дам) кому? чему?
    user3=десериализацию, % винительный (вижу) кого? что?
    user4=десериализацией, % творительный (доволен) кем? чем?
    user5=десериализации, % предложный (думаю) о ком? о чём?
}

\newglossaryentry{metaprogramming}{
    name=метапрограммирование,
    description={подход к разработке программного обеспечения, при
    котором создаются программы способные генерировать или преобразовывать
    другие программы (в том числе самих себя) как данные},
    user1=метапрограммирования, % родительный, (нет) кого, чего?
    user2=метапрограммированию, % дательный (дам) кому? чему?
    user3=метапрограммирование, % винительный (вижу) кого? что?
    user4=метапрограммированием, % творительный (доволен) кем? чем?
    user5=метапрограммировании, % предложный (думаю) о ком? о чём?
}

\newcommand{\glsgenitive}[1]{\glsuseri{#1}}   % ...... родительный, (нет) кого, чего?
\newcommand{\glsdative}[1]{\glsuserii{#1}}  % ........ дательный (дам) кому? чему?
\newcommand{\glsaccusative}[1]{\glsuseriii{#1}}  % ... винительный (вижу) кого? что?
\newcommand{\glsinstrumental}[1]{\glsuseriv{#1}} % ... творительный (доволен) кем? чем?
\newcommand{\glsprepositional}[1]{\glsuserv{#1}}  % ........... предложный (думаю) о ком? о чём?

%    user1=,  % родительный, (нет) кого, чего?
%    user2=,  % дательный (дам) кому? чему?
%    user3=,  % винительный (вижу) кого? что?
%    user4=,  % творительный (доволен) кем? чем?
%    user5=,  % предложный (думаю) о ком? о чём?

%\textbf{TeX} : Cистема компьютерной вёрстки, разработанная американским профессором информатики Дональдом Кнутом
%\textbf{панграмма} : Короткий текст, использующий все или почти все буквы алфавита
