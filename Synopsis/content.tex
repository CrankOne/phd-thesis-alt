\pdfbookmark{Общая характеристика работы}{characteristic}             % Закладка pdf
\section*{Общая характеристика работы}

\newcommand{\actuality}{\pdfbookmark[1]{Актуальность}{actuality}\underline{\textbf{\actualityTXT}}}
\newcommand{\progress}{\pdfbookmark[1]{Разработанность темы}{progress}\underline{\textbf{\progressTXT}}}
\newcommand{\aim}{\pdfbookmark[1]{Цели}{aim}\underline{{\textbf\aimTXT}}}
\newcommand{\tasks}{\pdfbookmark[1]{Задачи}{tasks}\underline{\textbf{\tasksTXT}}}
\newcommand{\aimtasks}{\pdfbookmark[1]{Цели и задачи}{aimtasks}\aimtasksTXT}
\newcommand{\novelty}{\pdfbookmark[1]{Научная новизна}{novelty}\underline{\textbf{\noveltyTXT}}}
\newcommand{\influence}{\pdfbookmark[1]{Практическая значимость}{influence}\underline{\textbf{\influenceTXT}}}
\newcommand{\methods}{\pdfbookmark[1]{Методология и методы исследования}{methods}\underline{\textbf{\methodsTXT}}}
\newcommand{\defpositions}{\pdfbookmark[1]{Положения, выносимые на защиту}{defpositions}\underline{\textbf{\defpositionsTXT}}}
\newcommand{\reliability}{\pdfbookmark[1]{Достоверность}{reliability}\underline{\textbf{\reliabilityTXT}}}
\newcommand{\probation}{\pdfbookmark[1]{Апробация}{probation}\underline{\textbf{\probationTXT}}}
\newcommand{\contribution}{\pdfbookmark[1]{Личный вклад}{contribution}\underline{\textbf{\contributionTXT}}}
\newcommand{\publications}{\pdfbookmark[1]{Публикации}{publications}\underline{\textbf{\publicationsTXT}}}

{\actuality} В последние десятилетия значительное внимание уделяется экспериментам,
направленным на поиск слабовзаимодействующих частиц. Как правило, такие
исследования требуют либо экстремальных энергий, либо накопления большой
статистики измерений в среднем энергетическом диапазоне. В последнем случае
широко применяются постановки с герметичным сбросом пучка, в которых
первичные частицы поглощаются массивной мишенью, а поиск редких процессов
производится на основе большой статистики наблюдений, в котором решающую роль
играют точность реконструкции физических величин и надёжность оценки
фоновых процессов.

Программные системы, сопровождающие физический эксперимент, разрабатываются
в условиях высокой динамики требований и ограниченного жизненного цикла.
После завершения исследовательской программы программный комплекс в целом
утрачивает актуальность, однако отдельные его компоненты сохраняют ценность
для повторного использования. Это обусловливает необходимость целенаправленной
стратегии разработки программного обеспечения и выделения архитектурных
инвариантов, в основе которых лежит ограниченный набор особенностей
организации эксперимента (например, использование триггерной системы).

\emph{Объектом} исследования, определяющим его актуальность, является разработка
специализированного программного обеспечения для реконструкции и анализа
данных в физическом эксперименте.

% степень проработанности
{\progress} В работах, посвящённых программному обеспечению для научных исследований
в целом~\cite{software-for-science-CarverEtAl2016, lifecycle-lenhardt2014data}
и автоматизации физических экспериментов в частности~\cite{Przedzinski2020PhD},
подчёркиваются его отличия от результатов промышленной разработки, требующие
глубокого погружения в предметную область и ограничивающее применимость
стандартных методов. Несмотря на значительные усилия по созданию общих решений
и программных сред~\cite{gaudi-framework-1, geant4-agostinelli, ROOT-framework, hep-roadmap-Albrecht2019},
специфика экспериментальной физики, связанная с
уникальностью экспериментальных постановок,
зачастую не позволяет полагаться на классические централизованные
каскадные подходы.
В результате исследовательские группы реализуют решения, исходя
из локальных задач и текущих потребностей, действуя в рамках
гибких методологий разработки и формируя таким образом собственную
программную базу. Экспертами отмечается, что это нередко
приводит к дублированию функциональности и затрудняет долговременное
сопровождение~\cite{elmer2018strategicplanscientificsoftware}.
В результате внедрение уже реализованных программ в новые задачи оказывается
менее оправданным, чем их повторная разработка. 
Тем не менее такой подход остаётся преобладающим и описывается
как практика <<ad hoc>> (или <<bottom-up>>~\cite{frameworks-Brun2011})
подразумевающая индуктивный переход от ситуативных решений
к более общим системам.

В качестве \emph{предмета} работы выступает создание
программного комплекса для реконструкции и анализа событий эксперимента
NA64 направленного на поиск новых слабовзаимодействующих частиц,
связывающих тёмный сектор с барионным веществом через портальные
взаимодействия. % На примере моделей описанных в работах ... % Характеристика работы по структуре во введении и в автореферате не отличается (ГОСТ Р 7.0.11, пункты 5.3.1 и 9.2.1), потому её загружаем из одного и того же внешнего файла, предварительно задав форму выделения некоторым параметрам

%Диссертационная работа была выполнена при поддержке грантов \dots

%\underline{\textbf{Объем и структура работы.}} Диссертация состоит из~введения,
%четырех глав, заключения и~приложения. Полный объем диссертации
%\textbf{ХХХ}~страниц текста с~\textbf{ХХ}~рисунками и~5~таблицами. Список
%литературы содержит \textbf{ХХX}~наименование.

\pdfbookmark{Содержание работы}{description}                          % Закладка pdf
\section*{Содержание работы}
Во \underline{\textbf{введении}} излагается актуальность
объекта исследований, проводимых в~рамках данной диссертационной работы,
%приводится обзор литературы по~изучаемой проблеме,
формулируется цель, ставятся задачи работы, излагается научная новизна
и практическая значимость представляемой работы.
%В~последующих главах
%сначала описывается общий принцип, позволяющий \dots, а~потом идёт
%апробация на частных примерах: \dots  и~\dots.


\underline{\textbf{Первая глава}} посвящена разработке
спецификаций, предложению и обоснованию архитектурных решений,
обеспечивающих варианты использования программного комплекса.

В разделе 1.1 описываются релевантные методологии разработки
программного обеспечения, излагаются основные требования к ним,
обосновываются черты гибридного подхода. Уточняется основная
терминология используемая в работе далее: виды полиморфизма,
точка расширения, архитектурные инварианты, использование
шаблонов C++ для реализации статического полиморфизма.
В соответствии с предложенной методологией в
разделе 1.2 фиксируется специфика предметной
области, включающую логический триггер и модель события,
а в разделе 1.3 кратко излагаются основные этапы жизненного
цикла эксперимента с точки зрения программного обеспечения,
включающие сценарии использования, сведённые в
иерархию, в основе которой лежат анализ, сопровождение набора данных
и процедуры калибровки детекторов. Подчёркивается
важность таких аспектов системы, как модульность,
детерминированность вычислений и изолированность компонент.
Аргументируется необходимость выведения модели события
и номенклатуры детекторов в точки расширения.
Раздел 1.4 завершает выведение основных архитектурных
ограничений и предлагает шаблоны
проектирования <<конвейер>>, <<издатель--подписчик>>
и <<фабрика>> в качестве механизма обеспечивающего
инвариантные свойства программного комплекса, а также
устанавливает основные качества коллекций в рамках
объектной модели события. Раздел 1.5 перечисляет существующие
программные решения --- общие окружения и инфраструктуры, а также
ряд популярных инструментов прикладного уровня.
Глава завершается разделом 1.6 содержащим итоговые
спецификации для организации программного комплекса.

% Вопрос №1: каким образом соотносится информация
%       о событии и его цифровая модель в смысле логического соответствия
%       статически-типизированным данным? (схемы, диаграммы, спецификации)

\underline{\textbf{Вторая глава}} посвящена описанию
эксперимента NA64 с целью получения предварительных численных
оценок и сведению обобщённых реализаций к конкретным программам
(инстанцированию шаблонов).
Раздел 2.1 описывает сигнальные процессы рождения $a$ и $A'$,
приводя лагранжиан взаимодействия, выражения для сечений и
распадной ширины. На основе приближения спектра эквивалентных
фотонов получены численные оценки для сигнального выхода
реакции $e^- + Z \rightarrow e^- + A' + Z$. В частности,
рассматривается сечение рождения $A'$:
\begin{comment}
%в соответствии с формулой \cite{bjorken}:
\[
\frac{1}{E^2_0} \frac{ d \sigma_{3 \rightarrow 2}}{d x d \cos{\theta_{A'}}} =
(8 \alpha \epsilon^2 \chi \beta_{A'}^2) \left[
    \frac{ 1 - x + x^2/2 }{U^2}
    + \frac{ (1-x)^2 m^2_{A'} }{U^4}
        \left( m^2_{A'} - \frac{U x}{1 - x} \right) \right],
\label{eq:bjorkenCS}
\]
где $x = E_0/E_{A'} \in (m_{A'}/E_0, 1 - m_{A'}/E_0)$ -- относительная энергия
$A'$, $\theta_{A'} \in [0, \pi]$ -- угол вылета, относительная скорость
$\beta = \sqrt{ 1 - m_{A'}/E_0 }$, и
$U = (E_0 \theta_{A'} )^2 x + m^2_{A'} (1-x)/x + m_e^2 x$.
Эффективный поток фотонов $\chi$ даётся интегральным выражением:
\begin{equation}
\chi = \int_{t_{min}}^{t_{max}} d t \frac{t - t_{min}}{t^2} G_2 (t),
\label{eq:photoFlux}
\end{equation}
где $G_2(t) = G_{2, el} + G_{2, inel}$ -- общий электромагнитный форм-фактор
\end{comment}
%в соответствии с формулой \cite{bjorken}:
\[
\frac{1}{E^2_0} \frac{ d \sigma_{2 \rightarrow 3}}{d x d \cos{\theta_{A'}}} \propto
\beta_{A'}^2 \left[
    \frac{ 1 - x + x^2/2 }{U^2}
    + \frac{ (1-x)^2 m^2_{A'} }{U^4}
        \left( m^2_{A'} - \frac{U x}{1 - x} \right) \right],
\label{eq:bjorkenCS}
\]
где $x = E_0/E_{A'}$ -- энергия $A'$ относительно энергии
налетающего лептона $E_0$,
$\theta_{A'}$ -- угол вылета,
$\beta = \sqrt{ 1 - m_{A'}/E_0 }$ и
$U = (E_0 \theta_{A'} )^2 x + m^2_{A'} (1-x)/x + m_e^2 x$.
Для дифференциального сечения такой формы вводятся следующие мажоранты:
\begin{equation*}
\begin{aligned}
    \mu_1(x, \theta) &= x^3 \theta \, 
        [E_0^2 \theta^2 x^2 + m_{A'}^2(1-x) + m_e^2 x^2]^{-2}, \\
    \mu_2(x) &= [2 E_0^2 \, (m_{A'}^2 (1 - x) + m_e^2 x ) ]^{-1},
\end{aligned}
\end{equation*}
используемые далее в методе обратной функции для построения
эффективного генератора псевдослучайных чисел с соответствующим
угловым и энергетическим распределением $A'$.
Приводятся оценки выхода сигнальных процессов для $A'$.
Описана аналогичная процедура для аксион-подобной частицы $a$.

Раздел 2.2 содержит краткое концептуальное описание
эксперимента NA64 и описывает общую номенклатуру
экспериментальных данных, имеющую значение для специализации обобщённых
реализаций программного комплекса.

Раздел 2.3 описывает основную номенклатуру
детекторов также являющуюся важным параметром в специализации
обобщённых реализаций. Приводятся описания основных
логических единиц триггерной системы: пучковых
счётчиков, калориметров и вето-детектора. Дано
краткое описание системы мечения первичных частиц,
организованной на основе магнитного спектрометра
и детектора синхротронного излучения. В перечне технических
параметров детектора особое значение имеет номенклатура
считывающей электроники, во многом определяющая
элементы объектной модели события.

Раздел 2.4 описывает сэмплирующие аналого-цифровые
преобразователи, применяемые для считывания сигналов
с фотоэлектронных умножителей и некоторых трековых
детекторов. Изложена проблема разрешения множественных
сигналов от близких по времени событий.

Раздел 2.5 содержит описание общей идеи метода CLs применяемого
для консервативной оценки чувствительности эксперимента.
В частности, на примере счётной статистики метод позволяет
строить исключённые области в пространстве
параметров сигнального процесса для заданного уровня
статистической значимости.

Глава завершается разделом 2.6 подводящим итоги главы
в виде описания функционального назначения элементов установки,
выделения общих и частных задач, соотносящихся с
систематизированными в предыдущей главе вариантами использования.

\underline{\textbf{Третья глава}} посвящена описанию реализации
программного комплекса и важным физическим результатам,
полученным с применением описанных в работе методов.

%картинку можно добавить так:
%\begin{figure}[ht]
%    \centerfloat{
%        \hfill
%        \subcaptionbox{\LaTeX}{%
%            \includegraphics[scale=0.27]{latex}}
%        \hfill
%        \subcaptionbox{Knuth}{%
%            \includegraphics[width=0.25\linewidth]{knuth1}}
%        \hfill
%    }
%    \caption{Подпись к картинке.}\label{fig:latex}
%\end{figure}
%Формулы в строку без номера добавляются так:
%\[
%    \lambda_{T_s} = K_x\frac{d{x}}{d{T_s}}, \qquad
%    \lambda_{q_s} = K_x\frac{d{x}}{d{q_s}},
%\]
%\underline{\textbf{Вторая глава}} посвящена исследованию
%\underline{\textbf{Третья глава}} посвящена исследованию
%Можно сослаться на свои работы в автореферате. Для этого в файле
%\verb!Synopsis/setup.tex! необходимо присвоить положительное значение
%счётчику \verb!\setcounter{usefootcite}{1}!. В таком случае ссылки на
%работы других авторов будут подстрочными.
%Изложенные в третьей главе результаты опубликованы в~\cite{vakbib1, vakbib2}.
%Использование подстрочных ссылок внутри таблиц может вызывать проблемы.
%В \underline{\textbf{четвертой главе}} приведено описание

\FloatBarrier
\pdfbookmark{Заключение}{conclusion}                                  % Закладка pdf
В \underline{\textbf{заключении}} приведены основные результаты работы, которые заключаются в следующем:
%% Согласно ГОСТ Р 7.0.11-2011:
%% 5.3.3 В заключении диссертации излагают итоги выполненного исследования, рекомендации, перспективы дальнейшей разработки темы.
%% 9.2.3 В заключении автореферата диссертации излагают итоги данного исследования, рекомендации и перспективы дальнейшей разработки темы.

\begin{comment}
\begin{enumerate}
  \item На основе анализа \ldots
  \item Численные исследования показали, что \ldots
  \item Математическое моделирование показало \ldots
  \item Для выполнения поставленных задач был создан \ldots
\end{enumerate}
\end{comment}

% 5.3.3 В заключении излагают итоги выполненного исследования,
%     рекомендации, перспективы дальнейшей разработки темы
%
% На мой взгляд, можно было бы использовать термин «эксперимент с фиксированной мишенью» в качестве обобщающего для применения разработанного ПК, добавляя в необходимых для этого местах уточнение «например, для эксперимента NA64». Упоминание этих конкретных условий и объекта станет очень уместным в соответствующих разделах диссертации, например, в разделе о практической применимости и внедрении результатов. Более того, напоминаю, что в Заключении, кроме основных выводов должны быть представлены перспективы и предложения по использованию полученных результатов – CERN и коллаборация здесь вообще «не сыграют», а скорее навредят. В Заключении автореферата сейчас перечислены, в основном, частные достижения, а они там не нужны; придётся переформулировать в обобщающие научные выводы, исключив заодно всякие «коллаборации» и прочие «уникальные условия».

Разработана и обоснована архитектура программного комплекса
предназначенного для реконструкции и анализа событий в
рамках задач возникающих в физическом эксперименте с триггерной
системой.
Предложенная архитектура реализует ограниченный набор
архитектурных инвариантов:
модульность,
потоковая обработка,
наличие объектной модели события,
детерминированность и идемпотентность вычислений.
Сведение этих принципов в программную архитектуру и их
обобщённая реализация являются оригинальным техническим
решением, при помощи которого, в рамках рассмотренной гибридной методологии
разработки, в работе на конкретных
примерах обеспечен полный цикл сопровождения эксперимента -- показаны этапы
моделирования, реконструкции, анализа и обработки данных. В частности:
\begin{enumerate}
    \item Обобщённая реализация конвейера обработки данных обеспеченного
    калибровочной информацией и усиленного набором встраиваемых
    искусственных языков, позволяет проводить реконструкцию
    и анализ экспериментальных событий, включая этапы калибровки и
    мониторинга во время набора данных,
    \item Предложенный генератор модельных событий для процесса
    фотообразования частицы $A'$ (тёмный фотон) на основе аналитических
    сечений позволяет эффективно моделировать
    класс реакций образования гипотетических частиц на тяжёлых ядрах,
    связанных со Стандартной моделью электромагнитным взаимодействием,
    \item Предложенный метод калибровки герметичных детекторов не
    нуждается в априорных оценках энерговыделения, и позволяет проводить
    перекалибровку детектора во время набора данных,
    \item Реализованная подсистема генерации машин конечных состояний для
    решения задач минимизации и численной аппроксимации позволяет динамически
    обуславливать различные сценарии отбора гипотез. В частности, обобщённая
    реализация алгоритма отыскания треков позволяет снизить комбинаторный фон,
    а алгоритм подгонки функции отклика сигналов с сэмплирующих
    амплитудно-цифровых преобразователей позволяет восстанавливать сигналы
    с использованием конкурирующих гипотез априорной формы
    импульса (частично компенсируя ограничения на частоту Найквиста).
\end{enumerate}

Практическая значимость предложенных решений:

\begin{itemize}
    \item Архитектурные инварианты и реализованные механизмы
    расширения программного комплекса нацелены на разработку
    в рамках коротких циклов, допускают упрощенную интеграцию с
    другими экспериментами построенными на триггерной логике (включающие
    иные калориметрические конфигурации, трековые подсистемы,
    системы сбора данных), снижая затраты на сопровождение
    и повторное использование программного кода, и отвечая таким образом
    основным задачам автоматизации физического эксперимента.
    \item Параметрическая реконструкция сигналов и связанная с ней
    свёрточная модель ливня повышают устойчивость извлечения
    признаков (временные и амплитудные характеристики)
    при разных частотах дискретизации и геометриях
    сэмплирующих калориметров
    \item Генератор с аналитическими сечениями ускоряет
    численные оценки фона/сигналов в сценариях поиска слабых
    сигналов по сравнению с более общими методами, не
    учитывающими форму.
\end{itemize}


%а также явная спецификация точек расширения в
%рамках обобщённых шаблонных реализаций конвейера данных.
% нельзя ли как-то связать с "гибридной методологией разработки"?


\pdfbookmark{Литература}{bibliography}                                % Закладка pdf
При использовании пакета \verb!biblatex! список публикаций автора по теме
диссертации формируется в разделе <<\publications>>\ файла
\verb!common/characteristic.tex!  при помощи команды \verb!\nocite!

\ifdefmacro{\microtypesetup}{\microtypesetup{protrusion=false}}{} % не рекомендуется применять пакет микротипографики к автоматически генерируемому списку литературы
\urlstyle{rm}                               % ссылки URL обычным шрифтом
\ifnumequal{\value{bibliosel}}{0}{% Встроенная реализация с загрузкой файла через движок bibtex8
    \renewcommand{\bibname}{\large \bibtitleauthor}
    \nocite{*}
    \insertbiblioauthor           % Подключаем Bib-базы
    %\insertbiblioexternal   % !!! bibtex не умеет работать с несколькими библиографиями !!!
}{% Реализация пакетом biblatex через движок biber
    % Цитирования.
    %  * Порядок перечисления определяет порядок в библиографии (только внутри подраздела, если `\insertbiblioauthorgrouped`).
    %  * Если не соблюдать порядок "как для \printbibliography", нумерация в `\insertbiblioauthor` будет кривой.
    %  * Если цитировать каждый источник отдельной командой --- найти некоторые ошибки будет проще.
    %
    %% authorvak
    \nocite{vakbib1}%
    \nocite{vakbib2}%
    \nocite{vakbib3}%
    \nocite{vakbib4}%
    \nocite{vakbib5}%
    \nocite{vakbib6}%
    \nocite{vakbib7}%
    \nocite{vakbib8}%
    \nocite{vakbib9}%
    \nocite{vakbib10}%
    \nocite{vakbib11}%
    \nocite{vakbib12}%
    %
    %% authorwos
    \nocite{wosbib1}%
    %
    %% authorscopus
    \nocite{scbib1}%
    %
    %% authorpatent
    \nocite{patbib1}%
    %
    %% authorprogram
    \nocite{progbib1}%
    %
    %% authorconf
    \nocite{confbib1}%
    \nocite{confbib2}%
    %
    %% authorother
    \nocite{bib1}%
    \nocite{bib2}%

    \ifnumgreater{\value{usefootcite}}{0}{
        \begin{refcontext}[labelprefix={}]
            \ifnum \value{bibgrouped}>0
                \insertbiblioauthorgrouped    % Вывод всех работ автора, сгруппированных по источникам
            \else
                \insertbiblioauthor      % Вывод всех работ автора
            \fi
        \end{refcontext}
    }{
        \ifnum \totvalue{citeexternal}>0
            \begin{refcontext}[labelprefix=A]
                \ifnum \value{bibgrouped}>0
                    \insertbiblioauthorgrouped    % Вывод всех работ автора, сгруппированных по источникам
                \else
                    \insertbiblioauthor      % Вывод всех работ автора
                \fi
            \end{refcontext}
        \else
            \ifnum \value{bibgrouped}>0
                \insertbiblioauthorgrouped    % Вывод всех работ автора, сгруппированных по источникам
            \else
                \insertbiblioauthor      % Вывод всех работ автора
            \fi
        \fi
        %  \insertbiblioauthorimportant  % Вывод наиболее значимых работ автора (определяется в файле characteristic во второй section)
        \begin{refcontext}[labelprefix={}]
            \insertbiblioexternal            % Вывод списка литературы, на которую ссылались в тексте автореферата
        \end{refcontext}
        % Невидимый библиографический список для подсчёта количества внешних публикаций
        % Используется, чтобы убрать приставку "А" у работ автора, если в автореферате нет
        % цитирований внешних источников.
        \printbibliography[heading=nobibheading, section=0, env=countexternal, keyword=biblioexternal, resetnumbers=true]%
    }
}
\ifdefmacro{\microtypesetup}{\microtypesetup{protrusion=true}}{}
\urlstyle{tt}                               % возвращаем установки шрифта ссылок URL
