{\actuality} В последние десятилетия значительное внимание уделяется экспериментам,
направленным на поиск слабовзаимодействующих частиц. Как правило, такие
исследования требуют либо экстремальных энергий, либо накопления большой
статистики измерений в среднем энергетическом диапазоне. В последнем случае
широко применяются постановки с герметичным сбросом пучка, в которых
первичные частицы поглощаются массивной мишенью, а поиск редких процессов
производится на основе большой статистики наблюдений, в котором решающую роль
играют точность реконструкции физических величин и надёжность оценки
фоновых процессов.

Программные системы, сопровождающие физический эксперимент, разрабатываются
в условиях высокой динамики требований и ограниченного жизненного цикла.
После завершения исследовательской программы программный комплекс в целом
утрачивает актуальность, однако отдельные его компоненты сохраняют ценность
для повторного использования. Это обусловливает необходимость целенаправленной
стратегии разработки программного обеспечения и выделения архитектурных
инвариантов, в основе которых лежит ограниченный набор особенностей
организации эксперимента (например, использование триггерной системы).

\emph{Объектом} исследования, определяющим его актуальность, является разработка
специализированного программного обеспечения для реконструкции и анализа
данных в физическом эксперименте.

% степень проработанности
{\progress} В работах, посвящённых программному обеспечению для научных исследований
в целом~\cite{software-for-science-CarverEtAl2016, lifecycle-lenhardt2014data}
и автоматизации физических экспериментов в частности~\cite{Przedzinski2020PhD},
подчёркиваются его отличия от результатов промышленной разработки, требующие
глубокого погружения в предметную область и ограничивающее применимость
стандартных методов. Несмотря на значительные усилия по созданию общих решений
и программных сред~\cite{gaudi-framework-1, geant4-agostinelli, ROOT-framework, hep-roadmap-Albrecht2019},
специфика экспериментальной физики, связанная с
уникальностью экспериментальных постановок,
зачастую не позволяет полагаться на классические централизованные
каскадные подходы.
В результате исследовательские группы реализуют решения, исходя
из локальных задач и текущих потребностей, действуя в рамках
гибких методологий разработки и формируя таким образом собственную
программную базу. Экспертами отмечается, что это нередко
приводит к дублированию функциональности и затрудняет долговременное
сопровождение~\cite{elmer2018strategicplanscientificsoftware}.
В результате внедрение уже реализованных программ в новые задачи оказывается
менее оправданным, чем их повторная разработка. 
Тем не менее такой подход остаётся преобладающим и описывается
как практика <<ad hoc>> (или <<bottom-up>>~\cite{frameworks-Brun2011})
подразумевающая индуктивный переход от ситуативных решений
к более общим системам.

В качестве \emph{предмета} работы выступает создание
программного комплекса для реконструкции и анализа событий эксперимента
NA64 направленного на поиск новых слабовзаимодействующих частиц,
связывающих тёмный сектор с барионным веществом через портальные
взаимодействия. % На примере моделей описанных в работах ...