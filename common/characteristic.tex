{\actuality} В последние десятилетия значительное внимание уделяется
экспериментам, направленным на поиск слабовзаимодействующих частиц,
чья природа выходит за рамки Стандартной модели. Как правило, такие
исследования требуют либо экстремальных энергий, либо накопления большой
статистики измерений в среднем энергетическом диапазоне. В последнем случае
широко применяются постановки с герметичным сбросом пучка, в которых
первичные частицы поглощаются массивной мишенью, а поиск редких событий
производится на основе большой статистики наблюдений, в котором решающее значение
играют точность реконструкции физических величин и надёжность оценки
фоновых процессов.

В то же время, средства автоматизации эксперимента,
во многом обеспечивающие требуемые точность и надёжность,
разрабатываются в условиях высокой динамики требований и
ограниченного жизненного цикла.
После завершения исследовательской программы они, как правило, утрачивают
актуальность как целостное решение, тогда как отдельные компоненты
могут представлять
ценность для повторного использования. Последнее обстоятельство
требует целенаправленной стратегии разработки программного обеспечения,
предполагающей повторное использование и интеграцию таких средств.
Руководящие ограничения такой стратегии опираются на ограниченный набор
особенностей, выбираемых на этапе проектирования информационных систем
с учётом специфики предметной области. Так например, использование триггерной
системы предполагает статистическую независимость физических
событий.

Объектом данного исследования является программное
обеспечения для реконструкции и анализа данных в физическом эксперименте.
В качестве предмета работы выступает создание
программного комплекса для реконструкции и анализа событий эксперимента
NA64\footnote{Эксперимент назван по внутренней номенклатуре CERN: <<North Area № 64>>}направленного
на поиск новых слабовзаимодействующих частиц,
%связывающих тёмный сектор с барионным веществом через портальные
%взаимодействия~\cite{NA64-first-paper},
из которых в работе рассматриваются
лёгкий аксион-подобный бозон~$a$~\cite{alps-PRD} и
т.н. тёмный фотон~$A'$~\cite{NA64-first-paper, DMG4-comphy-021}.

% степень проработанности
{\progress} В работах, посвящённых программному обеспечению для научных исследований
в целом %~\cite{software-for-science-CarverEtAl2016, lifecycle-lenhardt2014data}
и автоматизации физических экспериментов в частности %~\cite{Przedzinski2020PhD},
часто подчёркиваются отличия от промышленной разработки, требующие
глубокого погружения в предметную область и ограничивающее применимость
стандартных подходов к проектированию программного обеспечения, принятых в
индустрии. Несмотря на значительные усилия по созданию общих решений
и программных сред %~\cite{gaudi-framework-1, geant4-agostinelli, ROOT-framework, hep-roadmap-Albrecht2019},
специфика экспериментальной физики, связанная с
уникальностью экспериментальных постановок,
зачастую не позволяет полагаться на классические %централизованные
каскадные подходы.
В результате исследовательские группы реализуют решения, исходя
из текущих потребностей, действуя в рамках
гибких методологий разработки и формируя собственную программную базу
привязанную к локальным задачам. Экспертным сообществом отмечается,
что это нередко приводит к дублированию функциональности и затрудняет
долговременное сопровождение. %~\cite{elmer2018strategicplanscientificsoftware}.
В результате накопления технического долга перенос и внедрение
уже реализованных программ в новые
задачи оказывается зачастую менее оправданным, чем их повторная
реализация.
Тем не менее такой подход остаётся преобладающим и описывается
как общеупотребимая практика %<<ad hoc>> (или <<bottom-up>>~\cite{frameworks-Brun2011})
предполагающая индуктивный переход от ситуативных решений
к более общим системам.

{\aim} данной работы является разработка программного комплекса для
экспериментальной физики частиц обеспечивающего сопровождение
эксперимента на различных этапах его жизненного цикла, включая
моделирование, сопровождение набора данных, калибровку детекторов,
реконструкцию и анализ событий.
%в электромагнитных
%ливнях на основе параметрических моделей сигналов и ливней,
%а также аналитически управляемого моделирования фоновых/сигнальных
%процессов.

Для достижения поставленной цели, решаются следующие {\tasks}:
\begin{enumerate}[beginpenalty=10000]
    %\item Выделить типовые варианты использования  в рамках задач
    %автоматизации физического эксперимента.
    %\item Выделить архитектурные ограничения для программного комплекса,
    %обоснованные типовыми вариантами использования компьютерных
    %программ в рамках задач автоматизации физического эксперимента.
    \item Разработать спецификацию программной архитектуры,
    обеспечивающую основные задачи автоматизации физического эксперимента,
    в рамках архитектурных ограничений присущих экспериментам
    с логическим триггером.
    \item Реализовать архитектуру в виде
    обобщённых программ и алгоритмов, не подразумевающих специфику
    конкретного эксперимента и валидировать её посредством решения
    конкретных задач реконструкции и анализа событий NA64.
    \item Разработать и интегрировать в существующее специализированное
    программное окружение генератор событий для моделирования
    отклика установки на сигнальные события.
    \item Разработать и интегрировать систему генерации динамических
    машин состояния для выбора гипотез в задачах оптимизации, часто
    возникающих в рамках реконструкции физического события.
    \item Верифицировать предложенные реализации,
    применив к различным этапам жизненного цикла эксперимента,
    включающим Монте-Карло моделирование, сопровождение набора данных,
    калибровку детекторов, реконструкцию событий и анализ.
\end{enumerate}
При разработке и апробации программного комплекса необходимо
учесть, и, где это возможно, внедрять существующую алгоритмическую
базу.

{\methods}
С учётом сложившихся в науке практик создания компьютерных программ
прикладного уровня, целесообразно рассмотреть некоторые
подходы промышленной инженерии программного обеспечения,
учитывающие специфику предметной области и направленные
на инкрементную модель разработки. Таким требованиям вполне
отвечают т.н. гибкие (англ. \emph{agile})
методологии разработки, многие
из которых ставят приоритетом обеспечение высокой доступности
решений в рамках кратковременных циклов разработки.

На уровне конкретных технических средств, необходимо учесть
требования предъявляемые не только к длительности цикла разработки, но и
к быстродействию самих программ. Последнее условие значительно сужает
круг доступных выразительных средств, заставляя разработчиков
прибегать к сравнительно низкоуровневым решениям, в рамках которых
недостаток выразительности компенсируется за счёт высокой идиоматизации
(метапрограммирование, шаблоны, предметно-специфичные языки~\cite{perspectives-object-model-2012})
и инкапсуляции оптимизированных решений в рамках упрощённого
программного интерфейса.

Апробации реализованных технических решений осуществляются в рамках
конкретного эксперимента, чья постановка имеет глубокое методическое
обоснование. Она относится к семейству экспериментов с герметичным
сбросом пучка (англ. \emph{active beam-dump}). Методика измерений
включает поиск редких событий по сигнатуре недостающей энергии, а так
же регистрацию энерговыделения за массивным
поглотителем (англ. \emph{light shining through the wall}).
Статистическая интерпретация наблюдений формулируется
в рамках широко используемого в настоящее время метода CLs,
полно и непротиворечиво сформулированного в рамках частотной и байесовской
интерпретаций вероятности.

{\novelty}
\begin{enumerate}[beginpenalty=10000]
    \item Впервые для программных средств сопровождения
    физического эксперимента обосновано применение статического
    полиморфизма на основе объектных моделей
    %как базового механизма параметризации
    при синтезе программной архитектуры для последовательной
    обработки событий.
    \item Предложена эффективная алгоритмическая
    % вообще, алгоритмическая, но совет настаивает
    реализация генератора событий для симуляции
    рождения тёмного фотона в приближении спектра эквивалентных
    фотонов Вайцзеккера-Вильямса.
    %\item Разработана методика калибровки электромагнитного
    %калориметра не требующая полных априорных оценок
    %энерговыделения.
    %\item Подход, опирающийся на статический полиморфизм
    %специфичных для конкретного эксперимента объектных моделей
    %в качестве основного принципа выведения архитектуры применяется впервые.
    \item Впервые установлены верхние пределы на массу $m_a$ и константу
    смешивания $g_{a\gamma\gamma}$ в реакции $e^- + Z \to e^- + Z + a$
    для уровня статистической значимости $\alpha = 0{,}1$.
\end{enumerate}



%Разработана система автоматизации детектора КМД-2, включающая
%в себя комплекс программного обеспечения, предназначенного для
%выполнения следующих задач: чтения данных из электроники де-
%тектора; анализа данных в режиме реального времени и реализации
%на его основе третичного триггера и системы оперативного контроля
%качества данных; организации системы мониторинга детектора; организации
%системы управления детектором. Система автоматизации
%успешно проработала в течение всего срока эксплуатации детектора
%КМД-2.

{\defpositions}
\begin{enumerate}[beginpenalty=10000]
    % \item Разработано специализированное программное окружение для
    % решения ключевых задач реконструкции событий в экспериментах
    % физики высоких энергий, включая обработку данных трековых
    % детекторов и калориметров. Установлено соответствие архитектуры
    % данного окружения шаблону проектирования <<Конвейер>>~(Pipeline).
    \item Устройство модульного программного комплекса
    в парадигме обобщённого программирования
    для сопровождения полного
    цикла эксперимента по физике частиц,
    включая моделирование,
    накопление данных,
    калибровку детекторов,
    восстановление и анализ физических событий.
    %Комплекс реализован на основе статического полиморфизма,
    %в обобщённом виде и параметризуется ограниченным числом
    %параметров, специфичных для конкретного эксперимента
    %с триггерной системой, обеспечивая более короткие по сравнению
    %с существующими решениями циклы разработки.
    %По сравнению с существующими решениями обеспечивая
    %более короткие циклы разработки за счёт модульной структуры

    %\item Предложена эффективная алгоритмическая реализация генератора
    %событий для процессов фотообразования тёмной материи на основе
    %метода мажорирующих функций. Реализация интегрирована с моделями
    %генерации тёмных фотонов и лёгких аксионоподобных частиц.
    \item Эффективная алгоритмическая реализация генератора
    событий описывающего образование гипотетических
    частиц $A'$ (тёмный фотон) в реакциях рассеяния электронов
    на тяжёлых ядрах.
    
    \item Результат анализа экспериментальных данных NA64,
    соответствующих $2{,}84 \times 10^{11}$ электронов на мишени,
    %выполненный с использованием предложенных программных средств и методов,
    состоящий в том, что
    для реакции $e^{-} + Z \rightarrow e^{-} + Z + a$ 
    не обнаружено статистически значимого сигнала (при уровне значимости
    $\alpha = 0{,}1$) образования аксион-подобных частиц в
    диапазоне масс $m_a$ от~$10~\text{МэВ}$ до $100~\text{МэВ}$ и константы
    смешивания $g_{a \gamma \gamma}$ от~$2\times10^{-4}$ до~$10^{-2}$.
\end{enumerate}

{\contribution} Проектирование и реализация всех изложенных в работе
программ, за исключением описанных в
разделе <<Существующие программные решения>>, выполнены автором самостоятельно.

Алгоритмическая и программная реализация генератора~$A'$, лежащие в
основе \cite{DMG4-comphy-024, alps-PRD, DMG4-comphy-021}, выполнены
автором на основе существующих оценок сечения, в то время как внедрение и тестирование
генератора в интерфейсы окружения Geant4 выполнялись соавторами.

Обобщённая реализация алгоритма для поиска треков выполнена автором
на основе существующего описания алгоритма, с собственными дополнениями.
Тестирование и подстройка параметров выполнялись в сотрудничестве с
участниками коллаборации~NA64.

Автором предложен и реализованы метод реконструкции сигналов сэмплирующих
аналого-цифровых преобразователей
на основе подгонки функции отклика в конечной временной области и метод
калибровки электромагнитного калориметра.
Экспериментальная проверка алгоритма калибровки, настройка
алгоритма реконструкции сигналов, подбор параметров реконструкции трека
выполнялись в сотрудничестве с
участниками коллаборации~NA64.

Личный вклад автора в анализ данных ALP, опубликованный в \cite{alps-PRL}
состоит в статистическом обосновании критерия отбора фоновых событий,
моделировании и выделении исключённой области отсутствия сигнала на
основе анализа выполненного участниками коллаборации~NA64.

